% =====================================================================
% Systemdokumentation: AMR Low-Level Controller
% Fragment ohne Präambel (setzt Styles/Makros aus main.tex voraus)
% Version: 3.2.0 | Datum: 20.12.2025 | Status: Phase 1 abgeschlossen
% =====================================================================

\section{Systemdokumentation: AMR Low-Level Controller}
\label{sec:systemdokumentation-amr-llc}

\begin{infobox}[Meta]
\begin{itemize}
  \item Version: \textbf{3.2.0}
  \item Datum: \textbf{20.12.2025}
  \item Status: \(\checkmark\) \textbf{Phase 1 abgeschlossen}
\end{itemize}
\end{infobox}

\subsection{Architektur-Übersicht}

Das System implementiert eine \textbf{Hybrid-Echtzeit-Architektur}. Harte Echtzeit-Anforderungen (Motorsteuerung) sind strikt von Kommunikations-Aufgaben (micro-ROS) getrennt durch Dual-Core-Nutzung des ESP32-S3.

\subsubsection{Datenfluss (Übersicht)}

\begin{figure}[H]
  \centering
  \includegraphics[width=0.92\textwidth]{images/Architektur.png}
  \caption{Phase-1-Architektur (micro-ROS über USB-Serial): Der ESP32-S3 arbeitet als micro-ROS-Client und führt auf Core~0 die Regel-/Odometrie-Schleife (\(\SI{100}{\hertz}\)) inkl. Failsafe (\(\SI{2}{\second}\) Timeout) aus, während Core~1 die micro-ROS-Kommunikation bedient. Über USB-CDC bei \(\SI{921600}{\baud}\) ist der Raspberry~Pi~5 (Docker) mit dem \texttt{micro-ros-agent} verbunden; im ROS~2-Workspace werden Topics wie \topic{/cmd\_vel} (Motorsteuerung) und \topic{/odom\_raw} (Pose2D) getestet/überwacht.}
  \label{fig:architektur}
\end{figure}

\subsection{Firmware-Architektur (Dual-Core)}

Die Firmware nutzt \textbf{FreeRTOS}, um zwei parallele Tasks auf den physischen CPU-Kernen auszuführen.

\subsubsection{Core 0: \texttt{controlTask} (Hard Real-Time)}

\begin{table}[H]
\centering
\begin{tabularx}{\textwidth}{@{} l l @{}}
\toprule
\textbf{Eigenschaft} & \textbf{Wert} \\
\midrule
Task Name & \texttt{controlTask} \\
Frequenz & \(\SI{100}{\hertz}\) (deterministisch via \texttt{vTaskDelayUntil}) \\
Priorität & hoch (\texttt{configMAX\_PRIORITIES - 1}) \\
\bottomrule
\end{tabularx}
\end{table}

\textbf{Aufgaben:}
\begin{enumerate}
  \item Encoder-Interrupts atomar auslesen
  \item Odometrie integrieren (\(x, y, \theta\))
  \item Feedforward-Stellgrößen berechnen (Gain \(= 2.0\))
  \item Safety: Heartbeat-Timeout (\(2000\,\mathrm{ms}\)) \(\Rightarrow\) Not-Halt
\end{enumerate}

\subsubsection{Core 1: \texttt{loop()} (Communication)}

\begin{table}[H]
\centering
\begin{tabularx}{\textwidth}{@{} l l @{}}
\toprule
\textbf{Eigenschaft} & \textbf{Wert} \\
\midrule
Task & \texttt{loop()} (Arduino Standard) \\
Odom Publish & \(\SI{20}{\hertz}\) \\
Heartbeat & \(\SI{1}{\hertz}\) \\
\bottomrule
\end{tabularx}
\end{table}

\textbf{Aufgaben:}
\begin{enumerate}
  \item micro-ROS Executor Spin (Empfang/Versand)
  \item Serialisierung der DDS-Nachrichten
  \item I\textsuperscript{2}C-Kommunikation (geplant: IMU)
\end{enumerate}

\textbf{Datenaustausch:} über \texttt{SharedData} Struct, geschützt durch Mutex/Semaphore.

\subsubsection{Steuerungslogik (Feedforward, PID deaktiviert)}

\begin{lstlisting}[style=arduino]
// Feedforward + PID (PID aktuell deaktiviert)
float feedforward_gain = 2.0f;
float pwm_l = feedforward_gain * set_v_l + pid_left.compute(set_v_l, v_enc_l, dt);
float pwm_r = feedforward_gain * set_v_r + pid_right.compute(set_v_r, v_enc_r, dt);

// Begrenzen auf PWM-Bereich
pwm_l = constrain(pwm_l, -1.0f, 1.0f);
pwm_r = constrain(pwm_r, -1.0f, 1.0f);
\end{lstlisting}

\begin{infobox}[Hinweis]
PID ist deaktiviert (\(K_p=K_i=K_d=0\)), da die Encoder-Polarität invertiert ist. Feedforward ermöglicht stabile Open-Loop-Steuerung.
\end{infobox}

\subsection{ROS 2 Schnittstelle (API)}

\subsubsection{Topics}

\begin{table}[H]
\centering
\small
\setlength{\tabcolsep}{4pt}
\renewcommand{\arraystretch}{1.15}
\begin{tabularx}{\textwidth}{@{} l p{3.2cm} c c c X @{}}
\toprule
\textbf{Topic} & \textbf{Typ} & \textbf{Richtung} & \textbf{Frequenz} & \textbf{QoS} & \textbf{Beschreibung} \\
\midrule
\topic{/cmd\_vel} & \texttt{geometry\_msgs/Twist} & Sub & -- & Reliable & Geschwindigkeitsbefehle \\
\topic{/odom\_raw} & \texttt{geometry\_msgs/Pose2D} & Pub & \(\SI{20}{\hertz}\) & Best Effort & Odometrie (\(x,y,\theta\)) \\
\topic{/esp32/heartbeat} & \texttt{std\_msgs/Int32} & Pub & \(\SI{1}{\hertz}\) & Best Effort & Lebenszeichen \\
\topic{/esp32/led\_cmd} & \texttt{std\_msgs/Bool} & Sub & -- & Reliable & LED/MOSFET-Steuerung \\
\bottomrule
\end{tabularx}
\end{table}


\subsubsection{Nachrichtenformate}

\paragraph{\topic{/cmd\_vel} (Input)}
\begin{lstlisting}[style=shell]
linear:
  x: 0.15    # [m/s] Vorwärts (+) / Rückwärts (-)
  y: 0.0     # Nicht verwendet
  z: 0.0     # Nicht verwendet
angular:
  x: 0.0     # Nicht verwendet
  y: 0.0     # Nicht verwendet
  z: 0.5     # [rad/s] Links (+) / Rechts (-)
\end{lstlisting}

\paragraph{\topic{/odom\_raw} (Output)}
\begin{lstlisting}[style=shell]
x: 0.899     # [m]
y: -0.329    # [m]
theta: 6.09  # [rad]
\end{lstlisting}

\subsection{Konfiguration \& Parameter}

\subsubsection{\texttt{config.h}}

\begin{table}[H]
\centering
\begin{tabularx}{\textwidth}{@{} l l X @{}}
\toprule
\textbf{Parameter} & \textbf{Wert} & \textbf{Beschreibung} \\
\midrule
\texttt{LOOP\_RATE\_HZ} & 100 & Control-Zyklus (\(\SI{10}{\milli\second}\)) \\
\texttt{ODOM\_PUBLISH\_HZ} & 20 & Odom Publish (\(\SI{50}{\milli\second}\)) \\
\texttt{FAILSAFE\_TIMEOUT\_MS} & 2000 & Heartbeat-Timeout \\
\texttt{MOTOR\_PWM\_FREQ} & 20000 & \(\SI{20}{\kilo\hertz}\) (unhörbar) \\
\texttt{MOTOR\_PWM\_BITS} & 8 & Auflösung 0--255 \\
\texttt{PWM\_DEADZONE} & 35 & Mindest-PWM \\
\texttt{WHEEL\_DIAMETER} & 0.065 & Raddurchmesser \([\mathrm{m}]\) \\
\texttt{WHEEL\_BASE} & 0.178 & Spurbreite \([\mathrm{m}]\) \\
\bottomrule
\end{tabularx}
\end{table}

\subsubsection{PWM-Kanäle (getauscht für korrekte Richtung)}

\begin{lstlisting}[style=arduino]
#define PWM_CH_LEFT_A  1  // war 0
#define PWM_CH_LEFT_B  0  // war 1
#define PWM_CH_RIGHT_A 3  // war 2
#define PWM_CH_RIGHT_B 2  // war 3
\end{lstlisting}

\subsubsection{Regelung}

\begin{table}[H]
\centering
\begin{tabularx}{\textwidth}{@{} l l X @{}}
\toprule
\textbf{Parameter} & \textbf{Wert} & \textbf{Beschreibung} \\
\midrule
\texttt{PID\_KP} & 0.0 & deaktiviert \\
\texttt{PID\_KI} & 0.0 & deaktiviert \\
\texttt{PID\_KD} & 0.0 & deaktiviert \\
\texttt{feedforward\_gain} & 2.0 & Direkte Ansteuerung \\
\bottomrule
\end{tabularx}
\end{table}

\subsubsection{HAL-Pins (Übersicht)}

\begin{table}[H]
\centering
\begin{tabularx}{\textwidth}{@{} l l l X @{}}
\toprule
\textbf{Pin} & \textbf{Funktion} & \textbf{Modus} & \textbf{Hardware} \\
\midrule
D0 & Motor Left A & PWM \(\rightarrow\) CH 1 & Cytron MDD3A \\
D1 & Motor Left B & PWM \(\rightarrow\) CH 0 & Cytron MDD3A \\
D2 & Motor Right A & PWM \(\rightarrow\) CH 3 & Cytron MDD3A \\
D3 & Motor Right B & PWM \(\rightarrow\) CH 2 & Cytron MDD3A \\
D6 & Encoder Left & ISR (Rising) & JGA25-370 \\
D7 & Encoder Right & ISR (Rising) & JGA25-370 \\
D10 & LED/MOSFET & Digital Out & IRLZ24N \\
D4, D5 & I\textsuperscript{2}C & Wire & reserviert (MPU6050) \\
D8, D9 & Servo & PWM & reserviert (Kamera) \\
\bottomrule
\end{tabularx}
\end{table}

\subsection{Inbetriebnahme}

\subsubsection{Nach Pi Reboot}
\begin{lstlisting}[style=shell]
cd ~/amr-platform/docker
docker compose up -d
sleep 5
docker compose logs microros_agent --tail 5
\end{lstlisting}

\begin{cmdbox}[Erwartung]
\ttfamily running... \textbar\ fd: 3
\end{cmdbox}

\subsubsection{Nach ESP32 Reboot}
\begin{lstlisting}[style=shell]
cd ~/amr-platform/docker
docker compose restart microros_agent
sleep 5
docker compose logs microros_agent --tail 5
\end{lstlisting}

\subsubsection{Verifikation}
\paragraph{Topics prüfen}
\begin{lstlisting}[style=shell]
docker compose exec amr_dev bash
source /opt/ros/humble/setup.bash
ros2 topic list
\end{lstlisting}

\begin{cmdbox}[Erwartung]
\ttfamily
/cmd\_vel\\
/esp32/heartbeat\\
/esp32/led\_cmd\\
/odom\_raw\\
/parameter\_events\\
/rosout
\end{cmdbox}

\paragraph{Heartbeat prüfen}
\begin{lstlisting}[style=shell]
ros2 topic echo /esp32/heartbeat
\end{lstlisting}

\paragraph{Odometrie prüfen}
\begin{lstlisting}[style=shell]
ros2 topic echo /odom_raw --once
\end{lstlisting}

\paragraph{Motor-Test (Sicherheit)}
\begin{warnbox}[Sicherheit]
Motor-Tests nur mit aufgebockten Rädern (kein Bodenkontakt).
\end{warnbox}

\begin{lstlisting}[style=shell]
ros2 topic pub /cmd_vel geometry_msgs/msg/Twist \
  "{linear: {x: 0.15}, angular: {z: 0.0}}" -r 10
\end{lstlisting}

\subsection{Testergebnisse (20.12.2025)}

\begin{table}[H]
\centering
\begin{tabularx}{\textwidth}{@{} X X c @{}}
\toprule
\textbf{Test} & \textbf{Ergebnis} & \textbf{Status} \\
\midrule
Agent-Verbindung & \texttt{fd: 3} stabil & \(\checkmark\) \\
Heartbeat & \(\approx \SI{1}{\hertz}\) & \(\checkmark\) \\
Vorwärts & Räder drehen vorwärts & \(\checkmark\) \\
Rückwärts & Räder drehen rückwärts & \(\checkmark\) \\
Drehen links & Roboter dreht links & \(\checkmark\) \\
Drehen rechts & Roboter dreht rechts & \(\checkmark\) \\
Failsafe & Stop nach \(\approx \SI{2}{\second}\) & \(\checkmark\) \\
Odom & \(x,y,\theta\) plausibel & \(\checkmark\) \\
\bottomrule
\end{tabularx}
\end{table}

\subsection{Known Issues \& Lösungen}

\begin{table}[H]
\centering
\begin{tabularx}{\textwidth}{@{} X X X @{}}
\toprule
\textbf{Symptom} & \textbf{Ursache} & \textbf{Lösung} \\
\midrule
Roboter ruckelt & Failsafe greift & \texttt{FAILSAFE\_TIMEOUT\_MS} erhöhen (aktuell \(2000\,\mathrm{ms}\)) \\
Keine Odom-Daten & QoS Mismatch & Best Effort QoS nutzen \\
Motor reagiert nicht & Feedforward zu niedrig & \texttt{feedforward\_gain} erhöhen \\
PID eskaliert & Encoder-Polarität invertiert & PID deaktivieren (\(K_p=0\)) \\
Räder drehen falsch & PWM-Kanäle & A\(\leftrightarrow\)B tauschen in \texttt{config.h} \\
Topics fehlen & Agent nicht verbunden & \texttt{docker compose restart microros\_agent} \\
\bottomrule
\end{tabularx}
\end{table}

\subsection{Bekannte Einschränkungen}

\begin{enumerate}
  \item Open-Loop-Steuerung: PID deaktiviert, keine Geschwindigkeitsregelung
  \item Encoder A-only: Richtung aus Soll-Geschwindigkeit abgeleitet
  \item Odom-Rate: effektiv \(\SIrange{3}{6}{\hertz}\) durch Serial-Transport
\end{enumerate}

\subsection{Projektstruktur}

\begin{figure}[H]
  \centering
  \includegraphics[width=0.4\textwidth]{images/projekt-struktur.png}
  \caption{Projektstruktur des Repositories \texttt{amr-platform}: \texttt{firmware/} enthält die ESP32-S3-Firmware (PlatformIO mit \texttt{src/main.cpp} und \texttt{include/config.h}), \texttt{ros2\_ws/src/} den ROS~2-Workspace mit Paketen für Bridge/Bringup/Description sowie \texttt{sllidar\_ros2}. Die Laufzeitumgebung ist in \texttt{docker/} (Dockerfile, Compose, Entrypoint) gekapselt. Automatisierung und Betrieb liegen in \texttt{scripts/} (Deployment, micro-ROS-Agent-Service, Dokumentations-Converter). \texttt{docs/} bündelt Projektdokumentation; \texttt{README.md}, \texttt{LICENSE}, \texttt{start.html} und \texttt{main-design.css} bilden Einstieg und Styling der Doku.}
  \label{fig:projektstruktur}
\end{figure}


\subsection{Changelog}

\subsubsection{v3.2.0 (20.12.2025) -- Phase 1 Abschluss}
\begin{itemize}
  \item Motor-Richtung: PWM-Kanäle getauscht (A\(\leftrightarrow\)B)
  \item Steuerung: Feedforward (Gain \(= 2.0\)) statt PID
  \item PID: deaktiviert (\(K_p=K_i=K_d=0\)) wegen Encoder-Polarität
  \item Failsafe: Timeout auf \(2000\,\mathrm{ms}\) erhöht
  \item Tests: alle Richtungen validiert
\end{itemize}

\subsubsection{v3.1.0 (20.12.2025)}
\begin{itemize}
  \item Baudrate: \(921600\,\mathrm{Bd}\)
  \item PID: aktiviert (\(K_p = 1.0\))
  \item Problem: PID-Eskalation durch Encoder-Polarität
\end{itemize}

\subsubsection{v3.0.0 (14.12.2025) -- Major Release}
\begin{itemize}
  \item Architektur: Wechsel auf Dual-Core (App/Pro CPU Trennung)
  \item RTOS: FreeRTOS Tasks und Mutex-Synchronisation
  \item Daten: Optimierung auf \texttt{Pose2D} (Bandbreite reduziert)
  \item Hardware: Initialisierung aller Pins
\end{itemize}

\subsection{Nächste Schritte}

\begin{table}[H]
\centering
\begin{tabularx}{\textwidth}{@{} l X c @{}}
\toprule
\textbf{Phase} & \textbf{Beschreibung} & \textbf{Status} \\
\midrule
Phase 1 & micro-ROS ESP32-S3 & \(\checkmark\) \\
Phase 2 & Docker-Infrastruktur & \(\checkmark\) \\
Phase 3 & RPLidar A1 Integration & \(\square\) \\
Phase 4 & EKF Sensor Fusion & \(\square\) \\
Phase 5 & SLAM (\texttt{slam\_toolbox}) & \(\square\) \\
Phase 6 & Nav2 Autonome Navigation & \(\square\) \\
\bottomrule
\end{tabularx}
\end{table}
