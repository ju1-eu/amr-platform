% =============================================================================
% Phase 3 – RPLidar A1: Befehlsreferenz
% Fragment ohne Präambel (setzt Styles/Makros aus main.tex voraus)
% =============================================================================

\section{Phase 3: RPLidar A1 – Befehlsreferenz}
\label{sec:phase3-rplidar-befehle}

\begin{infobox}[Status]
\begin{itemize}
  \item Status: \textbf{abgeschlossen}
  \item Stand: \textbf{2025-12-20}
\end{itemize}
\end{infobox}

\subsection{Quick Start}

\subsubsection{1) Container starten (nach Pi Reboot)}
\begin{lstlisting}[style=shell]
cd ~/amr-platform/docker
docker compose up -d
docker compose ps
\end{lstlisting}

\subsubsection{2) RPLidar starten (Terminal 1)}
\begin{lstlisting}[style=shell]
docker compose exec amr_dev bash
source /opt/ros/humble/setup.bash
source /root/ros2_ws/install/setup.bash
ros2 launch sllidar_ros2 sllidar_a1_launch.py serial_port:=/dev/ttyUSB0
\end{lstlisting}

\subsubsection{3) Smoke-Tests (Terminal 2)}
\begin{lstlisting}[style=shell]
docker compose exec amr_dev bash
source /opt/ros/humble/setup.bash

# Topics prüfen
ros2 topic list

# Frequenz prüfen (~7.6 Hz)
ros2 topic hz /scan

# Daten prüfen
ros2 topic echo /scan --once
\end{lstlisting}

\subsection{Erwartete Topics}

\begin{cmdbox}[Topics]
\ttfamily
/cmd\_vel\\
/esp32/heartbeat\\
/esp32/led\_cmd\\
/odom\_raw\\
/scan\\
/parameter\_events\\
/rosout
\end{cmdbox}

\subsection{Scan-Daten Referenz}

\begin{table}[H]
\centering
\begin{tabularx}{\textwidth}{@{} l X @{}}
\toprule
\textbf{Feld} & \textbf{Erwartung} \\
\midrule
\texttt{frame\_id} & \texttt{laser} \\
\texttt{angle\_min} & \(-3.14\,\mathrm{rad}\) \\
\texttt{angle\_max} & \(+3.14\,\mathrm{rad}\) \\
\texttt{range\_min} & \(0.05\,\mathrm{m}\) \\
\texttt{range\_max} & \(12.0\,\mathrm{m}\) \\
Frequenz & \(\approx 7.6\,\mathrm{Hz}\) \\
\bottomrule
\end{tabularx}
\end{table}

\subsection{Troubleshooting}

\subsubsection{Fehler: \texttt{"Operation timeout"}}
\begin{lstlisting}[style=shell]
# Port blockiert? Prüfen:
sudo fuser /dev/ttyUSB0

# Falls belegt, Prozesse beenden:
sudo kill <PID>
\end{lstlisting}

\subsubsection{Device nicht vorhanden}
\begin{lstlisting}[style=shell]
ls -l /dev/ttyUSB*
dmesg | grep -i usb | tail -10
\end{lstlisting}

\subsubsection{Container sieht Device nicht}
\textbf{docker-compose.yml prüfen:}
\begin{lstlisting}[style=shell]
devices:
  - /dev/ttyUSB0:/dev/ttyUSB0
\end{lstlisting}

\subsection{Hardware-Info (RPLidar A1)}

\begin{table}[H]
\centering
\begin{tabularx}{\textwidth}{@{} l X @{}}
\toprule
\textbf{Parameter} & \textbf{Wert} \\
\midrule
Modell & RPLidar A1 \\
S/N & 74A5FA89C7E19EC8BCE499F0FF725670 \\
Firmware & 1.29 \\
Hardware Rev & 7 \\
Scan Mode & Sensitivity \\
Sample Rate & \(8\,\mathrm{kHz}\) \\
Scan Frequency & \(\approx 7.6\,\mathrm{Hz}\) \\
\bottomrule
\end{tabularx}
\end{table}
