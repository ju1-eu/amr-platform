% =====================================================================
% AMR Implementierungsplan – Vom Schaltplan zur autonomen Navigation
% Fragment ohne Präambel (setzt Styles/Makros aus main.tex voraus)
% Version: 2.0 | Stand: 2025-12-20 | Firmware: v3.2.0
% =====================================================================

\section{AMR Implementierungsplan}
\label{sec:amr-implementierungsplan}

\begin{infobox}[Meta]
\begin{itemize}
  \item Version: \textbf{2.0}
  \item Stand: \textbf{2025-12-20}
  \item Firmware: \textbf{v3.2.0}
\end{itemize}
\end{infobox}

\subsection{Grundprinzip: Vertikale Scheiben statt horizontaler Schichten}

Ein häufiger Fehler bei Robotik-Projekten ist ein \emph{Big-Bang-Integrationsversuch}: zuerst komplette Hardware, dann komplette Firmware, dann komplette Treiber --- und beim ersten Integrationstest ist alles gleichzeitig neu. Fehlersuche wird dann unverhältnismäßig teuer.

\textbf{Ansatz:} System in \emph{vertikale Scheiben} schneiden. Jede Phase liefert ein lauffähiges, testbares Teilsystem, bevor die nächste Komplexitätsstufe hinzukommt.

\begin{figure}[H]
  \centering
  \includegraphics[width=0.92\textwidth]{images/Grundprinzip.png}
  \caption{Grundprinzip „vertikale Scheiben“: Statt Hardware, Firmware, Wahrnehmung und Navigation separat „fertig“ zu bauen und erst am Ende im Big-Bang-Test zu integrieren, wird das System phasenweise als lauffähiger End-to-End-Strang erweitert. Jede Phase liefert ein testbares Teilsystem (z.\,B. Phase~1: micro-ROS + Motorsteuerung, Phase~2: Docker/ROS-Tooling, Phase~3: LiDAR-Scan), bevor mit EKF/Sensorfusion, SLAM und Nav2 die nächste Komplexitätsstufe hinzukommt.}
  \label{fig:grundprinzip-vertikale-scheiben}
\end{figure}


\subsection{Phasen-Übersicht}

\begin{table}[H]
\centering
\begin{tabularx}{\textwidth}{@{} l X l @{}}
\toprule
\textbf{Phase} & \textbf{Beschreibung} & \textbf{Status} \\
\midrule
Phase 1 & micro-ROS auf ESP32-S3 (USB-Serial) & \(\checkmark\) Abgeschlossen \\
Phase 2 & Docker-Infrastruktur & \(\checkmark\) Vorhanden \\
Phase 3 & RPLidar A1 Integration & \textbf{AKTUELL} \\
Phase 4 & EKF Sensor Fusion & \(\square\) \\
Phase 5 & SLAM (\texttt{slam\_toolbox}) & \(\square\) \\
Phase 6 & Nav2 Autonome Navigation & \(\square\) \\
\bottomrule
\end{tabularx}
\end{table}

% ---------------------------------------------------------------------
\subsection{Phase 1: micro-ROS auf ESP32-S3}
\label{subsec:amr-phase1}

\textbf{Ziel:} Native ROS~2 Kommunikation über USB-Serial mit Dual-Core FreeRTOS Architektur.

\textbf{Status:} \(\checkmark\) Abgeschlossen (2025-12-20) \textbar\ Firmware \textbf{v3.2.0}

\subsubsection{Architektur (Kurz)}

\begin{figure}[H]
  \centering
  \includegraphics[width=0.92\textwidth]{images/Architektur.png}
  \caption{Phase-1-Architektur (micro-ROS über USB-Serial): Der ESP32-S3 arbeitet als micro-ROS-Client und führt auf Core~0 die Regel-/Odometrie-Schleife (\(\SI{100}{\hertz}\)) inkl. Failsafe (\(\SI{2}{\second}\) Timeout) aus, während Core~1 die micro-ROS-Kommunikation bedient. Über USB-CDC bei \(\SI{921600}{\baud}\) ist der Raspberry~Pi~5 (Docker) mit dem \texttt{micro-ros-agent} verbunden; im ROS~2-Workspace werden Topics wie \topic{/cmd\_vel} (Motorsteuerung) und \topic{/odom\_raw} (Pose2D) getestet/überwacht.}
  \label{fig:architektur}
\end{figure}

\subsubsection{Topics}
\begin{table}[H]
\centering
\begin{tabularx}{\textwidth}{@{} l l l X @{}}
\toprule
\textbf{Topic} & \textbf{Typ} & \textbf{Richtung} & \textbf{Beschreibung} \\
\midrule
\topic{/cmd\_vel} & \texttt{geometry\_msgs/Twist} & Sub & Geschwindigkeitsbefehle \\
\topic{/odom\_raw} & \texttt{geometry\_msgs/Pose2D} & Pub & Odometrie (\(x,y,\theta\)) \\
\topic{/esp32/heartbeat} & \texttt{std\_msgs/Int32} & Pub & Lebenszeichen \\
\topic{/esp32/led\_cmd} & \texttt{std\_msgs/Bool} & Sub & LED-Steuerung \\
\bottomrule
\end{tabularx}
\end{table}

\subsubsection{Konfiguration (Kernwerte)}
\begin{table}[H]
\centering
\begin{tabularx}{\textwidth}{@{} l X @{}}
\toprule
\textbf{Parameter} & \textbf{Wert} \\
\midrule
Baudrate & \(921600\,\mathrm{Bd}\) \\
Feedforward Gain & \(2.0\) \\
PID & deaktiviert (\(K_p = 0\)) \\
Failsafe Timeout & \(2000\,\mathrm{ms}\) \\
PWM-Kanäle & getauscht (A\(\leftrightarrow\)B) \\
\bottomrule
\end{tabularx}
\end{table}

\subsubsection{Testergebnisse}
\begin{table}[H]
\centering
\begin{tabularx}{\textwidth}{@{} l c @{}}
\toprule
\textbf{Test} & \textbf{Status} \\
\midrule
Vorwärts & \(\checkmark\) \\
Rückwärts & \(\checkmark\) \\
Drehen links & \(\checkmark\) \\
Drehen rechts & \(\checkmark\) \\
Failsafe (2\,s) & \(\checkmark\) \\
Odom plausibel & \(\checkmark\) \\
\bottomrule
\end{tabularx}
\end{table}

\subsubsection{Cytron MDD3A – Dual-PWM (kritisch)}
\begin{warnbox}[Kritisch]
Der MDD3A verwendet \textbf{keinen} DIR-Pin, sondern \textbf{zwei PWM-Signale pro Motor}.
\end{warnbox}

\begin{table}[H]
\centering
\begin{tabularx}{\textwidth}{@{} r r X @{}}
\toprule
\textbf{M1A (PWM)} & \textbf{M1B (PWM)} & \textbf{Ergebnis} \\
\midrule
200 & 0 & Vorwärts \\
0 & 200 & Rückwärts \\
0 & 0 & Coast (Auslaufen) \\
\bottomrule
\end{tabularx}
\end{table}

% ---------------------------------------------------------------------
\subsection{Phase 2: Docker-Infrastruktur}
\label{subsec:amr-phase2}

\textbf{Ziel:} Container-basierte ROS~2 Umgebung für einfaches Deployment.

\textbf{Status:} \(\checkmark\) Vorhanden

\subsubsection{Container}
\begin{table}[H]
\centering
\begin{tabularx}{\textwidth}{@{} l l X @{}}
\toprule
\textbf{Container} & \textbf{Image} & \textbf{Funktion} \\
\midrule
\texttt{amr\_agent} & \texttt{microros/micro-ros-agent:humble} & Serial Agent \\
\texttt{amr\_dev} & Custom (ROS 2 Humble) & Workspace \\
\bottomrule
\end{tabularx}
\end{table}

\subsubsection{docker-compose.yml (Auszug)}
\begin{lstlisting}[style=shell]
services:
  microros_agent:
    image: microros/micro-ros-agent:humble
    container_name: amr_agent
    network_mode: host
    privileged: true
    restart: always
    command: serial --dev /dev/ttyACM0 -b 921600
    devices:
      - /dev/ttyACM0:/dev/ttyACM0

  amr_dev:
    build: .
    container_name: amr_base
    network_mode: host
    privileged: true
    volumes:
      - ../ros2_ws:/root/ros2_ws
    command: tail -f /dev/null
\end{lstlisting}

\subsubsection{Quick Start}
\begin{lstlisting}[style=shell]
cd ~/amr-platform/docker
docker compose up -d
docker compose exec amr_dev bash
source /opt/ros/humble/setup.bash
ros2 topic list
\end{lstlisting}

% ---------------------------------------------------------------------
\subsection{Phase 3: RPLidar A1 Integration}
\label{subsec:amr-phase3}

\textbf{Ziel:} \(360^\circ\) Laserscan für Umgebungswahrnehmung.

\textbf{Status:} aktuell (Port \filep{/dev/ttyUSB0} erkannt)

\subsubsection{Hardware}
\begin{table}[H]
\centering
\begin{tabularx}{\textwidth}{@{} l l l @{}}
\toprule
\textbf{Komponente} & \textbf{Port} & \textbf{Status} \\
\midrule
RPLidar A1 & \filep{/dev/ttyUSB0} & \(\checkmark\) erkannt \\
\bottomrule
\end{tabularx}
\end{table}

\subsubsection{Aufgaben}
\begin{itemize}[label=\(\square\)]
  \item \texttt{rplidar\_ros} Package installieren
  \item Launch-File erstellen
  \item \topic{/scan} verifizieren
  \item TF: \texttt{laser} \(\rightarrow\) \texttt{base\_link}
  \item RViz2 Visualisierung
\end{itemize}

\subsubsection{Geplante Topics}
\begin{table}[H]
\centering
\begin{tabularx}{\textwidth}{@{} l l X @{}}
\toprule
\textbf{Topic} & \textbf{Typ} & \textbf{Frequenz} \\
\midrule
\topic{/scan} & \texttt{sensor\_msgs/LaserScan} & \(\SIrange{5}{10}{\hertz}\) \\
\bottomrule
\end{tabularx}
\end{table}

\subsubsection{Launch (geplant)}
\begin{lstlisting}[style=shell]
ros2 launch rplidar_ros rplidar_a1_launch.py
\end{lstlisting}

\textbf{Meilenstein Phase 3:} \topic{/scan} publiziert, Daten in RViz2 sichtbar.

% ---------------------------------------------------------------------
\subsection{Phase 4: EKF Sensor Fusion}
\label{subsec:amr-phase4}

\textbf{Ziel:} Robuste Odometrie durch Fusion von Encoder-Daten (später + IMU).

\subsubsection{Aufgaben}
\begin{itemize}[label=\(\square\)]
  \item \texttt{robot\_localization} Package
  \item EKF Node konfigurieren
  \item \topic{/odom\_raw} \(\rightarrow\) \topic{/odometry/filtered}
  \item TF: \texttt{odom} \(\rightarrow\) \texttt{base\_link}
  \item Optional: IMU Integration (MPU6050)
\end{itemize}

\subsubsection{Geplante Topics}
\begin{table}[H]
\centering
\begin{tabularx}{\textwidth}{@{} l l X @{}}
\toprule
\textbf{Topic} & \textbf{Typ} & \textbf{Quelle} \\
\midrule
\topic{/odometry/filtered} & \texttt{nav\_msgs/Odometry} & EKF \\
\topic{/tf} & \texttt{tf2\_msgs/TFMessage} & EKF \\
\bottomrule
\end{tabularx}
\end{table}

\textbf{Meilenstein Phase 4:} TF-Baum korrekt, gefilterte Odometrie stabil.

% ---------------------------------------------------------------------
\subsection{Phase 5: SLAM (slam\_toolbox)}
\label{subsec:amr-phase5}

\textbf{Ziel:} Der Roboter baut eine Karte seiner Umgebung.

\subsubsection{Aufgaben}
\begin{itemize}[label=\(\square\)]
  \item \texttt{slam\_toolbox} konfigurieren
  \item Online Async SLAM
  \item Testraum kartieren
  \item Karte speichern (PGM + YAML)
\end{itemize}

\subsubsection{Launch}
\begin{lstlisting}[style=shell]
ros2 launch slam_toolbox online_async_launch.py params_file:=slam_params.yaml
\end{lstlisting}

\textbf{Meilenstein Phase 5:} Eine speicherbare Karte des Testraums existiert.

% ---------------------------------------------------------------------
\subsection{Phase 6: Nav2 Autonome Navigation}
\label{subsec:amr-phase6}

\textbf{Ziel:} Ziel auf Karte setzen, Roboter fährt autonom hin.

\subsubsection{Nav2 Stack (Überblick)}
\begin{table}[H]
\centering
\begin{tabularx}{\textwidth}{@{} l X @{}}
\toprule
\textbf{Komponente} & \textbf{Funktion} \\
\midrule
AMCL & Lokalisierung auf bekannter Karte \\
Planner Server & Globaler Pfad (A* / Dijkstra) \\
Controller Server & Lokale Hindernisvermeidung \\
Costmap & Hinderniskarte aus Sensordaten \\
BT Navigator & Verhaltenssteuerung \\
\bottomrule
\end{tabularx}
\end{table}

\textbf{Meilenstein Phase 6:} Roboter navigiert autonom, weicht Hindernissen aus.

% ---------------------------------------------------------------------
\subsection{Zukünftige Erweiterungen (optional)}

\subsubsection{Kamera \& AI}
\begin{itemize}
  \item IMX296 Global Shutter Kamera
  \item YOLOv8 auf Hailo-8L
  \item Personen-Erkennung \(\rightarrow\) Stopp-Verhalten
\end{itemize}

\subsubsection{PID-Regelung}
Aktuell: Feedforward (Open-Loop). Für präzisere Regelung:
\begin{itemize}
  \item Encoder-Polarität korrigieren (Quadratur-Encoder oder Richtungs-Heuristik verbessern)
  \item PID aktivieren (Beispielwerte aus früheren Tests: \(K_p=13.0\), \(K_i=5.0\), \(K_d=0.01\))
\end{itemize}

% ---------------------------------------------------------------------
\subsection{Hardware-Übersicht}

\begin{table}[H]
\centering
\begin{tabularx}{\textwidth}{@{} l X l @{}}
\toprule
\textbf{Komponente} & \textbf{Spezifikation} & \textbf{Status} \\
\midrule
Seeed XIAO ESP32-S3 & Dual-Core, USB-CDC & aktiv \\
Cytron MDD3A & Dual-PWM, \(\SIrange{4}{16}{\volt}\) & aktiv \\
JGA25-370 (2×) & \(\SI{12}{\volt}\) DC + Encoder & aktiv \\
Raspberry Pi 5 & 8\,GB, ROS~2 Humble & aktiv \\
RPLidar A1 & \(360^\circ\) 2D Lidar & erkannt \\
Hailo-8L & AI Accelerator & später \\
IMX296 & Global Shutter & später \\
MPU6050 & IMU (I\textsuperscript{2}C) & später \\
\bottomrule
\end{tabularx}
\end{table}

% ---------------------------------------------------------------------
\subsection{Risikomatrix}

\begin{table}[H]
\centering
\begin{tabularx}{\textwidth}{@{} X l l X @{}}
\toprule
\textbf{Risiko} & \textbf{Wahrscheinlichkeit} & \textbf{Impact} & \textbf{Status} \\
\midrule
micro-ROS inkompatibel & früher hoch & hoch & \(\checkmark\) gelöst \\
MDD3A-Ansteuerung & früher hoch & hoch & \(\checkmark\) Dual-PWM geklärt \\
PID-Eskalation & früher mittel & mittel & \(\checkmark\) Feedforward \\
Motor-Richtung falsch & früher mittel & mittel & \(\checkmark\) PWM getauscht \\
Failsafe greift zu früh & früher mittel & niedrig & \(\checkmark\) \(2000\,\mathrm{ms}\) \\
RPLidar-Treiber & niedrig & mittel & Phase 3 \\
Nav2-Tuning aufwändig & hoch & mittel & Zeit einplanen \\
\bottomrule
\end{tabularx}
\end{table}

% ---------------------------------------------------------------------
\subsection{Zeitplan (8 Wochen, grob)}

\begin{figure}[H]
  \centering
  \includegraphics[width=0.92\textwidth]{images/Zeitplan.png}
  \caption{Grobzeitplan über 8~Wochen: Die Roadmap ist in aufeinander aufbauende Phasen mit je ca. 2~Wochen Fokuszeit gegliedert. Phase~1 (micro-ROS) und Phase~2 (Docker/ROS-Tooling) sind abgeschlossen; Phase~3 (RPLidar: Scan-Erfassung und Stabilisierung) ist aktuell. Danach folgen Phase~4 (EKF/Sensorfusion), Phase~5 (SLAM: Karte + Lokalisierung) und Phase~6 (Nav2: Pfadplanung und autonome Navigation).}
  \label{fig:zeitplan-8wochen}
\end{figure}

% ---------------------------------------------------------------------
\subsection{Checkliste pro Phase}

Jede Phase ist erst abgeschlossen, wenn:
\begin{itemize}[label=\(\checkmark\)]
  \item die definierten Tests bestanden sind
  \item der Code committet und dokumentiert ist
  \item die Konfigurationsdateien versioniert sind
  \item ein kurzes Protokoll die Ergebnisse festhält
  \item der nächste Schritt klar ist
\end{itemize}

\textbf{Phase 1:} \(\checkmark\) alle Punkte erfüllt (2025-12-20) \\
\textbf{Phase 2:} \(\checkmark\) alle Punkte erfüllt

% ---------------------------------------------------------------------
\subsection{Changelog}

\subsubsection{v2.0 (2025-12-20)}
\begin{itemize}
  \item Phase 1: micro-ROS statt Serial-Bridge
  \item Firmware: v3.2.0 mit Feedforward
  \item Architektur: Dual-Core FreeRTOS
  \item Docker: Container-basiertes Deployment
  \item Phasen: reorganisiert (6 statt 7)
\end{itemize}

\subsubsection{v1.3 (2025-12-12)}
\begin{itemize}
  \item Phase 2 (Odometrie + PID) abgeschlossen
  \item Serial-Bridge Architektur (Legacy)
\end{itemize}

\begin{tipbox}[Merksatz]
Lieber weniger Features, die funktionieren, als viele Features, die zusammen crashen.
\end{tipbox}
