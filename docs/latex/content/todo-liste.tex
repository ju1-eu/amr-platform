% =============================================================================
% ToDo-Liste AMR-Projekt
% Fragment ohne Präambel (setzt Styles/Makros aus main.tex voraus)
% Stand: 2025-12-20 | Aktuelle Phase: 4 (URDF/TF/EKF)
% =============================================================================

\section{ToDo-Liste AMR-Projekt}
\label{sec:todo-liste-amr}

\begin{infobox}[Stand]
\begin{itemize}
  \item Stand: \textbf{2025-12-20}
  \item Aktuelle Phase: \textbf{Phase 4 (URDF/TF/EKF)}
\end{itemize}
\end{infobox}

\subsection{Phasen-Übersicht}

\begin{table}[H]
\centering
\begin{tabularx}{\textwidth}{@{} l X l @{}}
\toprule
\textbf{Phase} & \textbf{Beschreibung} & \textbf{Status} \\
\midrule
Phase 1 & micro-ROS auf ESP32-S3 & \(\checkmark\) Abgeschlossen \\
Phase 2 & Docker-Infrastruktur & \(\checkmark\) Abgeschlossen \\
Phase 3 & RPLidar A1 Integration & \(\checkmark\) Abgeschlossen \\
Phase 4 & URDF + TF + EKF & \(\triangleleft\) \textbf{NÄCHSTE} \\
Phase 5 & SLAM (\texttt{slam\_toolbox}) & \(\square\) \\
Phase 6 & Nav2 Autonome Navigation & \(\square\) \\
\bottomrule
\end{tabularx}
\end{table}

% -----------------------------------------------------------------------------
\subsection{Phase 1: micro-ROS ESP32-S3 (abgeschlossen)}

\subsubsection{Firmware v3.2.0 (2025-12-20)}
\begin{itemize}[label=\(\checkmark\)]
  \item micro-ROS Client über USB-CDC (Serial)
  \item Dual-Core FreeRTOS (Core 0: Control, Core 1: Comms)
  \item \topic{/cmd\_vel} \(\rightarrow\) Motorsteuerung
  \item \topic{/odom\_raw} \(\rightarrow\) Odometrie (\texttt{Pose2D})
  \item \topic{/esp32/heartbeat} \(\rightarrow\) Lebenszeichen (\(\SI{1}{\hertz}\))
  \item Failsafe (\(\SI{2000}{\milli\second}\) Timeout)
  \item Feedforward-Steuerung (Gain \(= 2.0\))
\end{itemize}

% -----------------------------------------------------------------------------
\subsection{Phase 2: Docker-Infrastruktur (abgeschlossen)}

\begin{itemize}[label=\(\checkmark\)]
  \item \texttt{amr\_agent} Container (micro-ROS Agent)
  \item \texttt{amr\_dev} Container (ROS 2 Humble Workspace)
  \item \texttt{docker compose up -d} funktioniert
  \item Volumes für \texttt{ros2\_ws} gemountet
\end{itemize}

% -----------------------------------------------------------------------------
\subsection{Phase 3: RPLidar A1 (abgeschlossen)}

\subsubsection{Verifiziert (2025-12-20)}
\begin{itemize}[label=\(\checkmark\)]
  \item \texttt{sllidar\_ros2} gebaut
  \item \topic{/scan} publiziert (\(\approx \SI{7.6}{\hertz}\))
  \item Scan-Daten plausibel (\(\SIrange{0.05}{12}{\meter}\))
  \item \texttt{frame\_id}: \texttt{laser}
  \item \texttt{docker-compose.yml} aktualisiert
\end{itemize}

\subsubsection{Hardware-Info}
\begin{table}[H]
\centering
\begin{tabularx}{\textwidth}{@{} l X @{}}
\toprule
\textbf{Parameter} & \textbf{Wert} \\
\midrule
S/N & 74A5FA89C7E19EC8BCE499F0FF725670 \\
Firmware & 1.29 \\
Scan Mode & Sensitivity \\
Sample Rate & \(\SI{8}{\kilo\hertz}\) \\
Frequenz & \(\approx \SI{7.6}{\hertz}\) \\
\bottomrule
\end{tabularx}
\end{table}

% -----------------------------------------------------------------------------
\subsection{Phase 4: URDF + TF + EKF (nächste Phase)}
\label{subsec:phase4-next}

\subsubsection{ToDos}
\begin{itemize}[label=\(\square\)]
  \item URDF erstellen
  \begin{itemize}[label=\(\square\)]
    \item \texttt{base\_footprint} (Boden)
    \item \texttt{base\_link} (Chassis)
    \item \texttt{laser} Frame
  \end{itemize}
  \item \texttt{robot\_state\_publisher} für statische TFs
  \item \texttt{odom\_converter.py} Bridge Node
  \begin{itemize}[label=\(\square\)]
    \item \topic{/odom\_raw} \(\rightarrow\) \topic{/odom}
    \item TF: \texttt{odom} \(\rightarrow\) \texttt{base\_footprint}
  \end{itemize}
  \item Optional: EKF (\texttt{robot\_localization})
\end{itemize}

\subsubsection{TF-Baum Ziel}
\begin{cmdbox}[TF-Ziel]
\begin{lstlisting}[style=shell]
odom -> base_footprint -> base_link -> laser
\end{lstlisting}
\end{cmdbox}

% -----------------------------------------------------------------------------
\subsection{Phase 5: SLAM}

\begin{itemize}[label=\(\square\)]
  \item \texttt{slam\_toolbox} installieren
  \item Online Async SLAM
  \item Testraum kartieren
  \item Karte speichern
\end{itemize}

% -----------------------------------------------------------------------------
\subsection{Phase 6: Nav2}

\begin{itemize}[label=\(\square\)]
  \item Nav2 Stack
  \item AMCL Lokalisierung
  \item Costmap
  \item Autonome Navigation
\end{itemize}

% -----------------------------------------------------------------------------
\subsection{Aktuelle Hardware-Ports}

\begin{table}[H]
\centering
\begin{tabularx}{\textwidth}{@{} l l X @{}}
\toprule
\textbf{Device} & \textbf{Port} & \textbf{Funktion} \\
\midrule
ESP32-S3 & \filep{/dev/ttyACM0} & micro-ROS (\(921600\,\mathrm{Bd}\)) \\
RPLidar A1 & \filep{/dev/ttyUSB0} & LaserScan (\topic{/scan}) \\
\bottomrule
\end{tabularx}
\end{table}

% -----------------------------------------------------------------------------
\subsection{Quick Reference}

\subsubsection{Container starten}
\begin{lstlisting}[style=shell]
cd ~/amr-platform/docker
docker compose up -d
\end{lstlisting}

\subsubsection{RPLidar starten}
\begin{lstlisting}[style=shell]
docker compose exec amr_dev bash
source /opt/ros/humble/setup.bash
source /root/ros2_ws/install/setup.bash
ros2 launch sllidar_ros2 sllidar_a1_launch.py serial_port:=/dev/ttyUSB0
\end{lstlisting}

\subsubsection{Topics prüfen}
\begin{lstlisting}[style=shell]
ros2 topic list
ros2 topic hz /scan
ros2 topic echo /scan --once
\end{lstlisting}
