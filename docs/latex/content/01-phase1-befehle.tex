% =============================================================================
% Phase 1 – Befehlsreferenz
% (Fragment ohne Präambel; setzt Listings/Boxen-Stile aus main.tex voraus)
% =============================================================================

\section{Phase 1 – Befehlsreferenz}
\label{sec:phase1-befehle}

\begin{infobox}[Zweck]
Schnelle Befehls-Sammlung für Phase~1 (micro-ROS Agent im Docker auf dem Pi, ESP32 per USB-Serial).
Arbeitsverzeichnis: \filep{\textasciitilde/amr-platform/docker}.
\end{infobox}

\subsection{Nach Pi Reboot}

\begin{lstlisting}[style=shell]
cd ~/amr-platform/docker
docker compose up -d
sleep 5
docker compose logs microros_agent --tail 5
\end{lstlisting}

\begin{cmdbox}[Erwartung]
\ttfamily running... \textbar\ fd: 3
\end{cmdbox}

\subsection{Nach ESP32 Reboot (USB umgesteckt)}

\begin{lstlisting}[style=shell]
cd ~/amr-platform/docker
docker compose restart microros_agent
sleep 5
docker compose logs microros_agent --tail 5
\end{lstlisting}

\begin{cmdbox}[Erwartung]
\ttfamily running... \textbar\ fd: 3
\end{cmdbox}

\subsection{Notfall-Stop (falls Räder drehen)}

\begin{warnbox}[Sicherheit]
Wenn sich der Rover unkontrolliert bewegt: \emph{sofort} Stop senden. Bei Motor-Tests Räder aufbocken.
\end{warnbox}

\begin{lstlisting}[style=shell]
docker compose exec amr_dev bash -c "source /opt/ros/humble/setup.bash && \
  ros2 topic pub --once /cmd_vel geometry_msgs/msg/Twist '{linear: {x: 0.0}, angular: {z: 0.0}}'"
\end{lstlisting}

\subsection{Smoke-Tests}

\subsubsection{1. In Container gehen}

\begin{lstlisting}[style=shell]
docker compose exec amr_dev bash
source /opt/ros/humble/setup.bash
\end{lstlisting}

\subsubsection{2. Topics prüfen}

\begin{lstlisting}[style=shell]
ros2 topic list
\end{lstlisting}

\begin{cmdbox}[Erwartung]
\ttfamily
/cmd\_vel\\
/esp32/heartbeat\\
/esp32/led\_cmd\\
/odom\_raw\\
/parameter\_events\\
/rosout
\end{cmdbox}

\subsubsection{3. Heartbeat prüfen}

\begin{lstlisting}[style=shell]
ros2 topic echo /esp32/heartbeat
\end{lstlisting}

\begin{tipbox}[Erwartung]
Counter inkrementiert ca. $1\times/\mathrm{s}$ (mit \texttt{Ctrl+C} beenden).
\end{tipbox}

\subsubsection{4. Odometrie prüfen}

\begin{lstlisting}[style=shell]
ros2 topic echo /odom_raw --once
\end{lstlisting}

\begin{tipbox}[Erwartung]
Ausgabe enthält plausible \texttt{x}, \texttt{y}, \texttt{theta} Werte.
\end{tipbox}

\subsubsection{5. Frequenzen messen}

\begin{lstlisting}[style=shell]
ros2 topic hz /esp32/heartbeat
\end{lstlisting}

\begin{lstlisting}[style=shell]
ros2 topic hz /odom_raw
\end{lstlisting}

\begin{tipbox}[Erwartung]
Heartbeat $\approx \SI{1}{\hertz}$, Odom $\approx \SIrange{3}{6}{\hertz}$.
\end{tipbox}

\subsection{Motor-Tests}

\begin{warnbox}[Sicherheit]
Räder aufbocken (kein Bodenkontakt). Notfall-Stop bereithalten.
\end{warnbox}

\subsubsection{Vorwärts}

\begin{lstlisting}[style=shell]
ros2 topic pub /cmd_vel geometry_msgs/msg/Twist \
  "{linear: {x: 0.15}, angular: {z: 0.0}}" -r 10
\end{lstlisting}

\subsubsection{Rückwärts}

\begin{lstlisting}[style=shell]
ros2 topic pub /cmd_vel geometry_msgs/msg/Twist \
  "{linear: {x: -0.15}, angular: {z: 0.0}}" -r 10
\end{lstlisting}

\subsubsection{Drehen links}

\begin{lstlisting}[style=shell]
ros2 topic pub /cmd_vel geometry_msgs/msg/Twist \
  "{linear: {x: 0.0}, angular: {z: 0.5}}" -r 10
\end{lstlisting}

\subsubsection{Drehen rechts}

\begin{lstlisting}[style=shell]
ros2 topic pub /cmd_vel geometry_msgs/msg/Twist \
  "{linear: {x: 0.0}, angular: {z: -0.5}}" -r 10
\end{lstlisting}

\subsubsection{Manueller Stop}

\begin{lstlisting}[style=shell]
ros2 topic pub --once /cmd_vel geometry_msgs/msg/Twist \
  "{linear: {x: 0.0}, angular: {z: 0.0}}"
\end{lstlisting}

\begin{tipbox}[Hinweis]
Nach \texttt{Ctrl+C} stoppt der Failsafe die Motoren automatisch nach ca. \SI{2}{\second}.
\end{tipbox}

\subsection{Failsafe-Test}

\begin{enumerate}
  \item Motor-Befehl senden (Räder drehen).
  \item \texttt{Ctrl+C} drücken.
  \item \SI{2}{\second} warten.
  \item \textbf{Erwartung:} Motoren stoppen automatisch.
\end{enumerate}

\subsection{Odom nach Fahrt prüfen}

\begin{lstlisting}[style=shell]
ros2 topic echo /odom_raw --once
\end{lstlisting}

\begin{tipbox}[Erwartung]
Nach Vorwärtsfahrt: \texttt{x} $> 0$. Nach Drehung: \texttt{theta} ändert sich.
\end{tipbox}

\subsection{Checkliste Phase 1}

\begin{table}[H]
\centering
\begin{tabularx}{\textwidth}{@{}lXc@{}}
\toprule
\textbf{Test} & \textbf{Erwartung} & \textbf{Status} \\
\midrule
Agent verbindet & \texttt{fd: 3} & $\square$ \\
Topics vorhanden & 6 Topics & $\square$ \\
Heartbeat & $\approx \SI{1}{\hertz}$ & $\square$ \\
Odom publiziert & \texttt{x, y, theta} Werte & $\square$ \\
Vorwärts & Räder drehen vorwärts & $\square$ \\
Rückwärts & Räder drehen rückwärts & $\square$ \\
Drehen links & Roboter dreht links & $\square$ \\
Drehen rechts & Roboter dreht rechts & $\square$ \\
Failsafe & Stop nach \SI{2}{\second} & $\square$ \\
\bottomrule
\end{tabularx}
\end{table}

\subsection{Troubleshooting}

\begin{table}[H]
\centering
\begin{tabularx}{\textwidth}{@{}lX@{}}
\toprule
\textbf{Problem} & \textbf{Lösung} \\
\midrule
\texttt{Serial port not found} & USB-Kabel prüfen, \texttt{ls /dev/ttyACM*} \\
Topics fehlen & \texttt{docker compose restart microros\_agent} \\
Räder reagieren nicht & Feedforward-Gain prüfen (2.0) \\
Falsches Verhalten & USB ziehen, Firmware prüfen \\
\bottomrule
\end{tabularx}
\end{table}
