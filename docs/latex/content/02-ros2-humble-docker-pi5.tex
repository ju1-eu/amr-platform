% =====================================================================
% Phase 2 – ROS 2 Humble auf Pi 5 via Docker + micro-ROS Agent
% Fragment ohne Präambel (setzt Styles/Makros aus main.tex voraus)
% =====================================================================

\section{Phase 2: ROS 2 Humble auf Raspberry Pi 5 (Docker) + micro-ROS Agent}
\label{sec:phase2-ros2-humble-docker-pi5}

\begin{infobox}[Status \& Version]
\begin{itemize}
  \item Status: \textbf{completed}
  \item Updated: \textbf{2025-12-20}
  \item Version: \textbf{2.0}
  \item Depends on: \textbf{Phase 1 (micro-ROS auf ESP32-S3)}
  \item Next: \textbf{Phase 3 (RPLidar A1)}
\end{itemize}
\end{infobox}

\subsection{Zielbild \& Definition of Done}

\subsubsection{Zielbild}
\begin{itemize}
  \item Raspberry Pi 5 (Raspberry Pi OS \textbf{64-bit}) betreibt \textbf{ROS 2 Humble} in \textbf{Docker}.
  \item micro-ROS Agent läuft reproduzierbar (Container) und verbindet sich über \textbf{USB-Serial} zum ESP32-S3.
  \item Host-ROS kann:
  \begin{itemize}
    \item \topic{/cmd\_vel} publizieren \(\rightarrow\) Motor reagiert
    \item \topic{/odom\_raw} empfangen \(\rightarrow\) Werte plausibel
    \item \topic{/esp32/heartbeat} empfangen \(\rightarrow\) Agent-Verbindung verifiziert
  \end{itemize}
\end{itemize}

\subsubsection{DoD (verifiziert 2025-12-20)}
\begin{itemize}[label=\(\checkmark\)]
  \item \texttt{docker compose up} startet Container ohne manuelle Nacharbeit.
  \item Agent verbindet stabil über \filep{/dev/ttyACM0} mit $921600\,\mathrm{Bd}$.
  \item ROS Smoke-Tests sind grün:
  \begin{itemize}[label=\(\checkmark\)]
    \item \texttt{ros2 topic list} zeigt \topic{/cmd\_vel}, \topic{/odom\_raw}, \topic{/esp32/heartbeat}, \topic{/esp32/led\_cmd}
    \item \texttt{ros2 topic pub /cmd\_vel ...} bewegt den Rover
    \item \texttt{ros2 topic echo /odom\_raw} liefert kontinuierliche Werte
  \end{itemize}
  \item Failsafe stoppt Motoren nach \SI{2}{\second} Timeout.
\end{itemize}

\subsection{Docker-Images (Regel)}

\begin{table}[H]
\centering
\begin{tabularx}{\textwidth}{@{} l l X @{}}
\toprule
\textbf{Service / Container} & \textbf{Image} & \textbf{Funktion} \\
\midrule
\texttt{microros\_agent} / \texttt{amr\_agent} & \texttt{microros/micro-ros-agent:humble} & Serial Agent \\
\texttt{amr\_dev} / \texttt{amr\_base} & Custom (ROS 2 Humble) & Workspace \\
\bottomrule
\end{tabularx}
\end{table}

\textbf{Warum Humble statt Jazzy:}
\begin{itemize}
  \item micro-ROS Agent für Humble stabiler auf \texttt{arm64}
  \item Kompatibilität mit bestehenden Packages (Nav2, \texttt{slam\_toolbox})
\end{itemize}

\subsection{Host-Voraussetzungen}

\subsubsection{System}
\begin{itemize}
  \item Raspberry Pi 5, Raspberry Pi OS \textbf{64-bit} (Bookworm)
  \item Docker Engine + Docker Compose Plugin
\end{itemize}

\subsubsection{USB-Serial prüfen}
\begin{lstlisting}[style=shell]
ls -l /dev/ttyACM*
# Erwartung: /dev/ttyACM0 (ESP32-S3)

ls -l /dev/ttyUSB*
# Erwartung: /dev/ttyUSB0 (RPLidar A1)
\end{lstlisting}

\subsection{Repo-Struktur}

\begin{figure}[H]
  \centering
  \includegraphics[width=0.4\textwidth]{images/projekt-struktur.png}
  \caption{Projektstruktur des Repositories \texttt{amr-platform}: \texttt{firmware/} enthält die ESP32-S3-Firmware (PlatformIO mit \texttt{src/main.cpp} und \texttt{include/config.h}), \texttt{ros2\_ws/src/} den ROS~2-Workspace mit Paketen für Bridge/Bringup/Description sowie \texttt{sllidar\_ros2}. Die Laufzeitumgebung ist in \texttt{docker/} (Dockerfile, Compose, Entrypoint) gekapselt. Automatisierung und Betrieb liegen in \texttt{scripts/} (Deployment, micro-ROS-Agent-Service, Dokumentations-Converter). \texttt{docs/} bündelt Projektdokumentation; \texttt{README.md}, \texttt{LICENSE}, \texttt{start.html} und \texttt{main-design.css} bilden Einstieg und Styling der Doku.}
  \label{fig:projektstruktur}
\end{figure}


\subsection{Docker Compose}

\subsubsection{docker-compose.yml}
\begin{lstlisting}[style=shell]
services:
  microros_agent:
    image: microros/micro-ros-agent:humble
    container_name: amr_agent
    network_mode: host
    privileged: true
    restart: always
    command: serial --dev /dev/ttyACM0 -b 921600
    devices:
      - /dev/ttyACM0:/dev/ttyACM0

  amr_dev:
    build: .
    container_name: amr_base
    network_mode: host
    privileged: true
    volumes:
      - ../ros2_ws:/root/ros2_ws
    command: tail -f /dev/null
\end{lstlisting}

\subsubsection{Dockerfile}
\begin{lstlisting}[style=shell]
FROM ros:humble-ros-base

RUN apt-get update && apt-get install -y \
    python3-colcon-common-extensions \
    ros-humble-tf2-tools \
    ros-humble-xacro \
    ros-humble-rviz2 \
    && rm -rf /var/lib/apt/lists/*

WORKDIR /root/ros2_ws
\end{lstlisting}

\subsection{Befehle}

\subsubsection{Nach Pi Reboot}
\begin{lstlisting}[style=shell]
cd ~/amr-platform/docker
docker compose up -d
sleep 5
docker compose logs microros_agent --tail 5
\end{lstlisting}

\begin{cmdbox}[Erwartung]
\ttfamily running... \textbar\ fd: 3
\end{cmdbox}

\subsubsection{Nach ESP32 Reboot}
\begin{lstlisting}[style=shell]
docker compose restart microros_agent
sleep 5
docker compose logs microros_agent --tail 5
\end{lstlisting}

\subsubsection{In Container gehen}
\begin{lstlisting}[style=shell]
docker compose exec amr_dev bash
source /opt/ros/humble/setup.bash
\end{lstlisting}

\subsection{Smoke-Tests}

\subsubsection{Topics prüfen}
\begin{lstlisting}[style=shell]
ros2 topic list
\end{lstlisting}

\begin{cmdbox}[Erwartung]
\ttfamily
/cmd\_vel\\
/esp32/heartbeat\\
/esp32/led\_cmd\\
/odom\_raw\\
/parameter\_events\\
/rosout
\end{cmdbox}

\subsubsection{Heartbeat}
\begin{lstlisting}[style=shell]
ros2 topic echo /esp32/heartbeat
\end{lstlisting}

\begin{tipbox}[Erwartung]
Counter steigt ca. $\SI{1}{\hertz}$.
\end{tipbox}

\subsubsection{Odometrie}
\begin{lstlisting}[style=shell]
ros2 topic echo /odom_raw --once
\end{lstlisting}

\begin{tipbox}[Erwartung]
Ausgabe enthält plausible \texttt{x}, \texttt{y}, \texttt{theta}.
\end{tipbox}

\subsubsection{Motor-Test (Räder aufbocken!)}
\begin{warnbox}[Sicherheit]
Motor-Tests nur mit aufgebockten Rädern (kein Bodenkontakt).
\end{warnbox}

\begin{lstlisting}[style=shell]
# Vorwärts
ros2 topic pub /cmd_vel geometry_msgs/msg/Twist \
  "{linear: {x: 0.15}, angular: {z: 0.0}}" -r 10

# Ctrl+C -> Failsafe stoppt nach 2s
\end{lstlisting}

\subsection{Troubleshooting}

\begin{table}[H]
\centering
\begin{tabularx}{\textwidth}{@{} X X X @{}}
\toprule
\textbf{Problem} & \textbf{Ursache} & \textbf{Lösung} \\
\midrule
Agent sieht ESP32 nicht & Device-Pfad falsch & \texttt{ls /dev/ttyACM*} prüfen \\
Topics fehlen & Agent nicht verbunden & \texttt{docker compose restart microros\_agent} \\
ROS sieht keine Topics & Domain-ID Mismatch & \texttt{network\_mode: host} nutzen \\
Motor reagiert nicht & Failsafe greift & Timeout prüfen (\SI{2000}{\milli\second}) \\
\bottomrule
\end{tabularx}
\end{table}

\subsection{Verifizierte Konfiguration}

\begin{table}[H]
\centering
\begin{tabularx}{\textwidth}{@{} l l @{}}
\toprule
\textbf{Parameter} & \textbf{Wert} \\
\midrule
ROS-Version & Humble \\
Agent-Image & \texttt{microros/micro-ros-agent:humble} \\
Baudrate & $921600\,\mathrm{Bd}$ \\
Device & \filep{/dev/ttyACM0} \\
Services/Container & \texttt{microros\_agent} (\texttt{amr\_agent}), \texttt{amr\_dev} (\texttt{amr\_base}) \\
Network & host \\
\bottomrule
\end{tabularx}
\end{table}

\subsection{Changelog}

\begin{table}[H]
\centering
\begin{tabularx}{\textwidth}{@{} l l X @{}}
\toprule
\textbf{Version} & \textbf{Datum} & \textbf{Änderungen} \\
\midrule
v2.0 & 2025-12-20 & Humble statt Jazzy, $921600\,\mathrm{Bd}$, Container-Namen aktualisiert, Status: abgeschlossen \\
v1.0 & 2025-12-19 & Initiale Jazzy-Version (überholt) \\
\bottomrule
\end{tabularx}
\end{table}
