% =====================================================================
% Phase 3 – RPLidar A1 Integration + /scan
% Fragment ohne Präambel (setzt Styles/Makros aus main.tex voraus)
% =====================================================================

\section{Phase 3: RPLidar A1 Integration (+ \topic{/scan})}
\label{sec:phase3-rplidar-a1}

\begin{infobox}[Status \& Version]
\begin{itemize}
  \item Status: \textbf{completed}
  \item Updated: \textbf{2025-12-20}
  \item Version: \textbf{3.0}
  \item Depends on: Phase 1 (micro-ROS ESP32-S3), Phase 2 (Docker ROS 2 Humble)
  \item Next: Phase 4 (URDF + TF + EKF)
\end{itemize}
\end{infobox}

\subsection{Zielbild \& Definition of Done}

\subsubsection{Zielbild}
\begin{itemize}
  \item RPLidar A1 läuft am Raspberry Pi 5 (Docker/ROS 2 Humble) als Laser-Treiber.
  \item Topic \topic{/scan} (\texttt{sensor\_msgs/msg/LaserScan}) ist stabil und liefert plausible Werte.
  \item \texttt{frame\_id} ist \texttt{laser} (TF-Anbindung an \texttt{base\_link} folgt in Phase 4).
\end{itemize}

\subsubsection{DoD (verifiziert 2025-12-20)}
\begin{itemize}[label=\(\checkmark\)]
  \item \texttt{ros2 topic hz /scan} zeigt stabile Frequenz \(\approx \SI{7.6}{\hertz}\)
  \item \texttt{ros2 topic echo /scan --once} liefert plausible Werte
  \item \texttt{frame\_id}: \texttt{laser}
  \item Scan-Range: \(-\pi\) bis \(+\pi\), \(\SIrange{0.05}{12.0}{\meter}\)
\end{itemize}

\subsection{Hardware-Info (verifiziert)}

\subsubsection{Sensor}
\begin{table}[H]
\centering
\begin{tabularx}{\textwidth}{@{} l X @{}}
\toprule
\textbf{Parameter} & \textbf{Wert} \\
\midrule
Modell & RPLidar A1 \\
S/N & 74A5FA89C7E19EC8BCE499F0FF725670 \\
Firmware & 1.29 \\
Hardware Rev & 7 \\
Scan Mode & Sensitivity \\
Sample Rate & \(\SI{8}{\kilo\hertz}\) \\
\bottomrule
\end{tabularx}
\end{table}

\subsubsection{Gemessene Werte}
\begin{table}[H]
\centering
\begin{tabularx}{\textwidth}{@{} l X @{}}
\toprule
\textbf{Parameter} & \textbf{Wert} \\
\midrule
Frequenz & \(\approx \SI{7.6}{\hertz}\) \\
\texttt{range\_min} & \(\SI{0.05}{\meter}\) \\
\texttt{range\_max} & \(\SI{12.0}{\meter}\) \\
\texttt{angle\_min} & \(-3.14\,\mathrm{rad}\) \\
\texttt{angle\_max} & \(+3.14\,\mathrm{rad}\) \\
\bottomrule
\end{tabularx}
\end{table}

\subsubsection{Anschluss}
\begin{table}[H]
\centering
\begin{tabularx}{\textwidth}{@{} l l X @{}}
\toprule
\textbf{Device} & \textbf{Port} & \textbf{Hinweis} \\
\midrule
RPLidar A1 & \filep{/dev/ttyUSB0} & USB-Adapter: \texttt{cp210x} \\
\bottomrule
\end{tabularx}
\end{table}

\subsection{Docker-Integration}

\begin{warnbox}[Wichtig]
Kein separater \texttt{rplidar}-Container. Das erzeugt Port-Konflikte. RPLidar wird im \texttt{amr\_dev}-Container gestartet.
\end{warnbox}

\subsubsection{docker-compose.yml (relevanter Auszug)}
\begin{lstlisting}[style=shell]
services:
  microros_agent:
    image: microros/micro-ros-agent:humble
    container_name: amr_agent
    network_mode: host
    privileged: true
    restart: always
    command: serial --dev /dev/ttyACM0 -b 921600
    devices:
      - /dev/ttyACM0:/dev/ttyACM0

  amr_dev:
    build: .
    container_name: amr_base
    network_mode: host
    privileged: true
    stdin_open: true
    tty: true
    volumes:
      - ../ros2_ws:/root/ros2_ws
    devices:
      - /dev/ttyUSB0:/dev/ttyUSB0
    command: tail -f /dev/null
\end{lstlisting}

\subsection{Installation}

\subsubsection{Repository klonen (im Container)}
\begin{lstlisting}[style=shell]
cd ~/amr-platform/docker
docker compose exec amr_dev bash

cd /root/ros2_ws/src
git clone https://github.com/Slamtec/sllidar_ros2.git
\end{lstlisting}

\subsubsection{Build}
\begin{lstlisting}[style=shell]
cd /root/ros2_ws
source /opt/ros/humble/setup.bash
colcon build --packages-select sllidar_ros2
source install/setup.bash
\end{lstlisting}

\subsection{RPLidar starten \& Smoke-Tests}

\subsubsection{Terminal 1: Lidar-Node starten}
\begin{lstlisting}[style=shell]
cd ~/amr-platform/docker
docker compose exec amr_dev bash
source /opt/ros/humble/setup.bash
source /root/ros2_ws/install/setup.bash
ros2 launch sllidar_ros2 sllidar_a1_launch.py serial_port:=/dev/ttyUSB0
\end{lstlisting}

\begin{cmdbox}[Erwartete Log-Infos (Auszug)]
\ttfamily
SLLidar S/N: 74A5FA89C7E19EC8BCE499F0FF725670\\
Firmware Ver: 1.29\\
Hardware Rev: 7\\
health status : OK\\
current scan mode: Sensitivity, sample rate: 8 Khz, max\_distance: 12.0 m
\end{cmdbox}

\subsubsection{Terminal 2: Smoke-Tests}
\begin{lstlisting}[style=shell]
cd ~/amr-platform/docker
docker compose exec amr_dev bash
source /opt/ros/humble/setup.bash

ros2 topic list
ros2 topic hz /scan
ros2 topic echo /scan --once
\end{lstlisting}

\subsubsection{Erwartete Topics}
\begin{cmdbox}[Topic-Liste]
\ttfamily
/cmd\_vel\\
/esp32/heartbeat\\
/esp32/led\_cmd\\
/odom\_raw\\
/scan\\
/parameter\_events\\
/rosout
\end{cmdbox}

\subsubsection{Prüfpunkte für \topic{/scan}}
\begin{table}[H]
\centering
\begin{tabularx}{\textwidth}{@{} l X @{}}
\toprule
\textbf{Feld} & \textbf{Erwartung} \\
\midrule
\texttt{header.frame\_id} & \texttt{laser} \\
\texttt{angle\_min}/\texttt{angle\_max} & \(-3.14\,\mathrm{rad}\) bis \(+3.14\,\mathrm{rad}\) \\
\texttt{range\_min}/\texttt{range\_max} & \(\SI{0.05}{\meter}\) bis \(\SI{12.0}{\meter}\) \\
\texttt{ranges[]} & Werte \(\SIrange{0.05}{12.0}{\meter}\), \texttt{inf} bei keiner Reflexion \\
Frequenz & \(\approx \SI{7.6}{\hertz}\) \\
\bottomrule
\end{tabularx}
\end{table}

\subsection{Troubleshooting}

\subsubsection{\texttt{"Operation timeout"}}
\textbf{Typisch:} Port ist von anderem Prozess belegt.
\begin{lstlisting}[style=shell]
# Auf Pi Host prüfen:
sudo fuser /dev/ttyUSB0

# Falls belegt:
sudo kill <PID>
\end{lstlisting}

\subsubsection{\topic{/scan} nicht vorhanden}
\begin{itemize}
  \item Lidar-Node muss in Terminal 1 weiterlaufen.
  \item Wenn mit \texttt{Ctrl+C} gestoppt: kein \topic{/scan}.
\end{itemize}

\subsubsection{Device nicht vorhanden}
\begin{lstlisting}[style=shell]
ls -l /dev/ttyUSB*
dmesg | grep -i usb | tail -10
\end{lstlisting}

\subsubsection{Container sieht Device nicht}
\begin{lstlisting}[style=shell]
# docker-compose.yml:
devices:
  - /dev/ttyUSB0:/dev/ttyUSB0
\end{lstlisting}

\subsubsection{Baudrate falsch}
Standard ist \(\SI{115200}{\baud}\). Falls Timeout:
\begin{lstlisting}[style=shell]
ros2 launch sllidar_ros2 sllidar_a1_launch.py \
  serial_port:=/dev/ttyUSB0 serial_baudrate:=256000
\end{lstlisting}

\subsection{Statischer TF (temporär für RViz2)}
Bis Phase 4 (URDF/TF/EKF) fertig ist:
\begin{lstlisting}[style=shell]
ros2 run tf2_ros static_transform_publisher \
  0.12 0.0 0.15 0 0 0 base_link laser
\end{lstlisting}

\subsection{Nächste Schritte (Phase 4)}

\begin{enumerate}
  \item URDF erstellen inkl. Transform \texttt{base\_link} \(\rightarrow\) \texttt{laser}
  \item \texttt{robot\_state\_publisher} für TF-Baum
  \item \texttt{odom\_converter.py} Bridge Node (Pose2D \(\rightarrow\) Odometry/TF)
  \item Optional: EKF (robot\_localization)
\end{enumerate}

\subsection{Changelog}

\begin{table}[H]
\centering
\begin{tabularx}{\textwidth}{@{} l l X @{}}
\toprule
\textbf{Version} & \textbf{Datum} & \textbf{Änderungen} \\
\midrule
v3.0 & 2025-12-20 & Status: completed; kein separater RPLidar-Container; Hardware-Info verifiziert; Troubleshooting erweitert \\
v2.0 & 2025-12-20 & Humble statt Jazzy \\
v1.0 & 2025-12-19 & Initiale Version \\
\bottomrule
\end{tabularx}
\end{table}
